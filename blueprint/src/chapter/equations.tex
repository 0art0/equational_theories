\chapter{Selected laws}\label{subgraph-eq}

In this project we study the 4694 laws (up to symmetry and relabeling) of total order at most $4$.

Selected laws of interest are listed below, as well as in \href{https://github.com/teorth/equational_theories/blob/main/equational_theories/Equations.lean}{this file}.

\begin{definition}[Equation 1]\label{eq1}\lean{Equation1}\leanok\uses{magma-def}  Equation 1 is the law $0 \formaleq 0$ (or the equation $x=x$).
\end{definition}

This is the trivial law, satisfied by all magmas. It is self-dual.


\begin{definition}[Equation 2]\label{eq2}\lean{Equation2}\leanok\uses{magma-def}  Equation 2 is the law $0 \formaleq 1$ (or the equation $x=y$).
\end{definition}

This is the singleton law, satisfied only by the empty and singleton magmas.  It is self-dual.

\begin{definition}[Equation 3]\label{eq3}\lean{Equation3}\leanok\uses{magma-def}  Equation 3 is the law $0 \formaleq 0 \op 0$ (or the equation $x = x \op x$).
\end{definition}

This is the idempotence law.  It is self-dual.

\begin{definition}[Equation 4]\label{eq4}\lean{Equation4}\leanok\uses{magma-def}  Equation 4 is the law $0 \formaleq 0 \op 1$ (or the equation $x = x \op y$).
\end{definition}

This is the left absorption law.

\begin{definition}[Equation 5]\label{eq5}\lean{Equation5}\leanok\uses{magma-def}  Equation 5 is the law $0 \formaleq 1 \op 0$ (or the equation $x = y \op x$).
\end{definition}

This is the right absorption law (the dual of \Cref{eq4}).

\begin{definition}[Equation 6]\label{eq6}\lean{Equation6}\leanok\uses{magma-def}  Equation 6 is the law $0 \formaleq 1 \op 1$ (or the equation $x = y \op y$).
\end{definition}

This law is equivalent to the singleton law.

\begin{definition}[Equation 7]\label{eq7}\lean{Equation7}\leanok\uses{magma-def}  Equation 7 is the law $0 \formaleq 1 \op 2$ (or the equation $x = y \op z$).
\end{definition}

This law is equivalent to the singleton law.

\begin{definition}[Equation 8]\label{eq8}\lean{Equation8}\leanok\uses{magma-def}  Equation 8 is the law $0 \formaleq 0 \op (0 \op 0)$ (or the equation $x = x \op (x \op x)$).
\end{definition}

\begin{definition}[Equation 14]\label{eq14}\lean{Equation14}\leanok\uses{magma-def}  Equation 14 is the law $0 \formaleq  1 \op (0 \op 1)$ (or the equation $x = y \op (x \op y))$.
\end{definition}

Appears in Problem A1 from Putnam 2001.

\begin{definition}[Equation 16]\label{eq16}\lean{Equation16}\leanok\uses{magma-def}  Equation 16 is the law $0 \formaleq  1 \op (1 \op 0)$ (or the equation $x = y \op (y \op x))$.
\end{definition}

\begin{definition}[Equation 23]\label{eq23}\lean{Equation23}\leanok\uses{magma-def}  Equation 23 is the law $0 \formaleq  (0 \op 0) \op 0$ (or the equation $x = (x \op x) \op x$).
\end{definition}

This is the dual of \Cref{eq8}.

\begin{definition}[Equation 29]\label{eq29}\lean{Equation29}\leanok\uses{magma-def}  Equation 29 is the law $0 \formaleq  (1 \op 0) \op 1$ (or the equation $x = (y \op x) \op y)$.
\end{definition}

Appears in Problem A1 from Putnam 2001.  Dual to \Cref{eq14}.

\begin{definition}[Equation 38]\label{eq38}\lean{Equation38}\leanok\uses{magma-def}  Equation 38 is the law $0 \op 0  \formaleq  0 \op 1$ (or the equation $x \op x = x \op y$).
\end{definition}

This law asserts that the magma operation is independent of the second argument.

\begin{definition}[Equation 39]\label{eq39}\lean{Equation39}\leanok\uses{magma-def}  Equation 39 is the law $0 \op 0  \formaleq  1 \op 0$ (or the equation $x \op x = y \op x$).
\end{definition}

This law asserts that the magma operation is independent of the first argument (the dual of \Cref{eq38}).

\begin{definition}[Equation 40]\label{eq40}\lean{Equation40}\leanok\uses{magma-def}  Equation 40 is the law $0 \op 0  \formaleq  1 \op 1$ (or the equation $x \op x = y \op y$).
\end{definition}

This law asserts that all squares are constant. It is self-dual.

\begin{definition}[Equation 41]\label{eq41}\lean{Equation41}\leanok\uses{magma-def}  Equation 41 is the law $0 \op 0  \formaleq  1 \op 2$ (or the equation $x \op x = y \op z$).
\end{definition}

This law is equivalent to the constant law, \Cref{eq46}.

\begin{definition}[Equation 42]\label{eq42}\lean{Equation42}\leanok\uses{magma-def}  Equation 42 is the law $0 \op 1  \formaleq  0 \op 2$ (or the equation $x \op y = x \op z$).
\end{definition}

Equivalent to \Cref{eq38}.

\begin{definition}[Equation 43]\label{eq43}\lean{Equation43}\leanok\uses{magma-def}  Equation 43 is the law $0 \op 1  \formaleq  1 \op 0$ (or the equation $x \op y = y \op x$).
\end{definition}

The commutative law. It is self-dual.

\begin{definition}[Equation 45]\label{eq45}\lean{Equation45}\leanok\uses{magma-def}  Equation 45 is the law $0 \op 1  \formaleq  2 \op 1$ (or the equation $x \op y = z \op y$).
\end{definition}

This is the dual of \Cref{eq42}.

\begin{definition}[Equation 46]\label{eq46}\lean{Equation46}\leanok\uses{magma-def}  Equation 46 is the law $0 \op 1  \formaleq  2 \op 3$ (or the equation $x \op y = z \op w$).
\end{definition}

The constant law: all products are constant. It is self-dual.

\begin{definition}[Equation 168]\label{eq168}\lean{Equation168}\leanok\uses{magma-def}  Equation 168 is the law $0  \formaleq  (1 \op 0) \op (0 \op 2)$ (or the equation $x = (y \op x) \op (x \op z)$).
\end{definition}

The law of a central groupoid. It is self-dual.

\begin{definition}[Equation 381]\label{eq381}\lean{Equation381}\leanok\uses{magma-def}  Equation 381 is the law $0 \op 1  \formaleq  (0 \op 2) \op 1$ (or the equation $x \op y = (x \op z) \op y$).
\end{definition}

Appears in Putnam 1978, Problem A4, part (b).

\begin{definition}[Equation 387]\label{eq387}\lean{Equation387}\leanok\uses{magma-def}  Equation 387 is the law $0 \op 1  \formaleq  (1 \op 1) \op 0$ (or the equation $x \op y = (y \op y) \op x$).
\end{definition}

Introduced in \href{https://mathoverflow.net/a/450905/766}{MathOverflow}.

\begin{definition}[Equation 477]\label{eq477}\lean{Equation477}\leanok\uses{magma-def}  Equation 477 is the law $0 \formaleq 1 \op (0 \op (1 \op (1 \op 1)))$ (or the equation $x = y \op (x \op (y \op (y \op y)))$).
\end{definition}

\begin{definition}[Equation 953]\label{eq953}\lean{Equation953}\leanok\uses{magma-def}  Equation 953 is the law $0 = 1 \op ((2 \op 0) \op (2 \op 2))$ (or the equation $x = y \op ((z \op x) \op (z \op z))$).
\end{definition}

\begin{definition}[Equation 1571]\label{eq1571}\lean{Equation1571}\leanok\uses{magma-def}  Equation 1571 is the law $0 \formaleq  (1 \op 2) \op (1 \op (0 \op 2))$ (or the equation $x = (y \op z) \op (y \op (x \op z))$).
\end{definition}

Introduced in \cite{mendelsohn-padmanabhan}.

\begin{definition}[Equation 1689]\label{eq1689}\lean{Equation1689}\leanok\uses{magma-def}  Equation 1689 is the law $0 \formaleq  (1 \op 0) \op ((0 \op 2) \op 2)$ (or the equation $x = (y \op x) \op ((x \op z) \op z)$).
\end{definition}

Mentioned in \cite{Kisielewicz2}.

\begin{definition}[Equation 2662]\label{eq2662}\lean{Equation2662}\leanok\uses{magma-def}  Equation 2662 is the law $0 \formaleq  ((0 \op 1) \op (0 \op 1)) \op 0$ (or the equation $x = ((x \op y) \op (x \op y)) \op x$).
\end{definition}

Appears in \cite{mendelsohn-padmanabhan}.

\begin{definition}[Equation 3722]\label{eq3722}\lean{Equation3722}\leanok\uses{magma-def}  Equation 3722 is the law $0 \op 1  \formaleq  (0 \op 1) \op (0 \op 1)$ (or the equation $x \op y = (x \op y) \op (x \op y)$).
\end{definition}

Appears in Putnam 1978, Problem A4, part (a).  It is self-dual.

\begin{definition}[Equation 3744]\label{eq3744}\lean{Equation3744}\leanok\uses{magma-def}  Equation 3744 is the law $0 \op 1  \formaleq  (0 \op 2) \op (3 \op 1)$ (or the equation $x \op y = (x \op z) \op (w \op y)$).
\end{definition}

This law is called a ``bypass operation'' in Putnam 1978, Problem A4. It is self-dual.

\begin{definition}[Equation 4512]\label{eq4512}\lean{Equation4512}\leanok\uses{magma-def}  Equation 4512 is the law $0 \op (1 \op 2)  \formaleq  (0 \op 1) \op 2$ (or the equation $x \op (y \op z) = (x \op y) \op z$).
\end{definition}

The associative law. It is self-dual.

\begin{definition}[Equation 4513]\label{eq4513}\lean{Equation4513}\leanok\uses{magma-def}  Equation 4513 is the law $0 \op (1 \op 2)  \formaleq  (0 \op 1) \op 3$ (or the equation $x \op (y \op z) = (x \op y) \op w$).
\end{definition}

\begin{definition}[Equation 4522]\label{eq4522}\lean{Equation4522}\leanok\uses{magma-def}  Equation 4522 is the law $0 \op (1 \op 2)  \formaleq  (0 \op 3) \op 4$ (or the equation $x \op (y \op z) = (x \op w) \op u$).
\end{definition}

Dual to \Cref{eq4579}.

\begin{definition}[Equation 4564]\label{eq4564}\lean{Equation4564}\leanok\uses{magma-def}  Equation 4564 is the law $0 \op (1 \op 2)  \formaleq  (3 \op 1) \op 2$ (or the equation $x \op (y \op z) = (w \op y) \op z$).
\end{definition}

Dual to \Cref{eq4513}.

\begin{definition}[Equation 4579]\label{eq4579}\lean{Equation4579}\leanok\uses{magma-def}  Equation 4579 is the law $0 \op (1 \op 2)  \formaleq  (3 \op 4) \op 2$ (or the equation $x \op (y \op z) = (w \op u) \op z$).
\end{definition}

Dual to \Cref{eq4522}.

\begin{definition}[Equation 4582]\label{eq4582}\lean{Equation4582}\leanok\uses{magma-def}  Equation 4582 is the law $0 \op (1 \op 2)  \formaleq  (3 \op 4) \op 5$ (or the equation $x \op (y \op z) = (w \op u) \op v$).
\end{definition}

This law asserts that all triple constants (regardless of bracketing) are constant.
