An invariant is generally a \emph{syntactic} property of an expression. However, invariants can also be described and calculated \emph{semantically} through the notion of a \emph{lifting magma family}, described below. The general idea is that the value of an invariant for an expression can be computed by substituting specific values for the variables in the expression and evaluating the result in a certain magma in the lifting magma family; additional requirements ensure that the fundamental property of invariants is satisfied.

\begin{definition}[Lifting Magma Family]\label{lifting-magma-family}
A \emph{lifting magma family} is a family of magmas $\{G_\alpha\}$, one for each type $\alpha$, satisfying the following properties:
\begin{itemize}
\item For each type $\alpha$, there is a function $\iota_\alpha : \alpha \to G_\alpha$.
\item Given a function $f : \alpha \to G_\alpha$, there is a magma homomorphism $\operatorname{lift}{f} : G_\alpha \to G_\alpha$ such that $\operatorname{lift}{f}(\iota_\alpha(x)) = f(x)$ for all $x$ in $\alpha$.
\end{itemize}
\end{definition}

\begin{example}
The free Abelian groups form a lifting magma family. When the underlying set is finite, the groups are isomorphic to $\mathbb{Z}^n$.
\end{example}

\begin{example}
Lists form a lifting magma family.
\end{example}

The key consequence of the definition \ref{lifting-magma-family} is that it is significantly easier to check whether an equation is satisfied in a lifting magma family.

\begin{theorem}[Evaluation theorem for lifting magma families]\label{lifting-magma-basis-evaluation}
Suppose $E$ is an equation involving a set of variables $X$, and let $G$ be a lifting magma family.

Determining whether $E$ is satisfied by $G_X$ is equivalent to checking that $E$ is true with the specific substitution $\iota_X$.

\end{theorem}
\begin{proof}
For the forward direction, suppose $E$ is satisfied by $G_X$. Then, by definition, any substitution of the variables in $E$ with elements of $G_X$ will yield a true equation. In particular, substituting according to $\iota_X$ will yield a true equation.

For the reverse direction, suppose that $E$ is true when evaluated with the substitution $\iota_X$. Now, consider an arbitrary substitution of variables $f : X \to G_X$. By the lifting magma family property, there is a magma homomorphism $\operatorname{lift}{f} : G_X \to G_X$ such that $\operatorname{lift}{f}(\iota_X(x)) = f(x)$ for all $x$ in $X$. In other words, applying the substitution $f$ is equivalent to first applying to substitution $\iota_X$ and then applying the homomorphism $\operatorname{lift}{f}$. Since $E$ is true when evaluated with the substitution $\iota_X$, it is also true after applying the homomorphism $\operatorname{lift}{f}$. Thus, $E$ is satisfied by $G_X$.
\end{proof}

\begin{theorem}[The fundamental property of invariants]\label{fundamental-property-of-invariants}
Let $E$ and $E'$ be equations involving a set of variables $X$, and let $G$ be a lifting magma family.

If $E$ is true with the substitution $\iota_X$, and $E$ implies $E'$, then so is $E'$.
\end{theorem}
\begin{proof}
Applying the evaluation theorem \ref{lifting-magma-basis-evaluation}, we see that $E$ is satisfied by $G_X$. Since $E$ implies $E'$, $E'$ is also satisfied by $G_X$, and in particular, $E'$ is true with the substitution $\iota_X$.
\end{proof}

\begin{remark}
The result of evaluating an expression along the function $\iota_X : X \to G_X$ \emph{is} the invariant.

In the case of Abelian groups, the result of evaluation is the variables in the expression with multiplicity.
In the case of lists, the result of evaluation is the variables in the expression in the order they appear.

When the lifting magma family has good computational properties, calculating the invariant becomes easy.
\end{remark}

\begin{remark}
Given an equation $\phi$ in the language of magmas (possibly involving logical operations other than equality and universal quantification), the initial (i.e., most general) magmas satisfying $\phi$ (provided they exist) form a lifting magma family.

However, for the purpose of generating invariants, we are interested in lifting magma families with convenient descriptions that are computationally tractable.
\end{remark}

\begin{remark}
Suppose $S$ is a finite set of equations in the language of magmas that is a confluent term rewriting system under a certain ordering of the terms (in the sense of the Knuth-Bendix algorithm). Then the initial magmas satisfying $S$ form a lifting magma family where equality of elements in the magma is decidable.

This offers a way of generating examples of lifting magma families with good compuational properties for computing invariants of expressions.
\end{remark}
