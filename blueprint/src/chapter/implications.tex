\chapter{Subgraph implications}

Interesting implications between the subgraph equations in Chapter \ref{subgraph-eq}. To reduce clutter, trivial or very easy implications will not be displayed here.

\begin{theorem}[387 implies 43]\label{387_implies_43}\uses{eq387,eq43}\lean{Subgraph.Equation387_implies_Equation43}\leanok  Definition \ref{eq387} implies Definition \ref{eq43}.
\end{theorem}

\begin{proof}\leanok (From \href{https://mathoverflow.net/a/450905/766}{MathOverflow}).
  By Definition \ref{eq387}, one has the law
\begin{equation}\label{387-again}
  (x \circ x) \circ y = y \circ x.
\end{equation}
Specializing to $y=x \circ x$, we conclude
$$(x \circ x) \circ (x \circ x) = (x \circ x) \circ x$$
and hence by another application of \eqref{eq387} we see that $x \circ x$ is idempotent:
\begin{equation}\label{idem}
  (x \circ x) \circ (x \circ x) = x \circ x.
\end{equation}
Now, replacing $x$ by $x \circ x$ in \eqref{387-again} and then using \eqref{idem} we see that
$$ (x \circ x) \circ y = y \circ (x \circ x)$$
so in particular $x \circ x$ commutes with $y \circ y$:
\begin{equation}\label{op-idem} (x \circ x) \circ (y \circ y) = (y \circ y) \circ (x \circ x).
\end{equation}
Also, from two applications of \eqref{387-again} one has
$$(x \circ x) \circ (y \circ y) = (y \circ y) \circ x = x \circ y.$$
Thus \eqref{op-idem} simplifies to $x \circ y = y \circ x$, which is Definition \ref{eq43}.
\end{proof}
