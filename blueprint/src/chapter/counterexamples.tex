\chapter{Subgraph counterexamples}

Some counterexamples for the anti-implications between the subgraph equations in Chapter \ref{subgraph-eq}.

\begin{theorem}[46 does not imply 4]\label{46_not_imply_4}\lean{Subgraph.Equation46_not_implies_Equation4}\leanok\uses{eq46,eq4} Definition \ref{eq46} does not imply Definition \ref{eq4}.
\end{theorem}

\begin{proof}\leanok Use the natural numbers $\N$ with operation $x \circ y := 0$.
\end{proof}

\begin{theorem}[4 does not imply 4582]\label{4_not_imply_4582}\lean{Subgraph.Equation4_not_implies_Equation4582}\leanok\uses{eq4,eq4582} Definition \ref{eq4} does not imply Definition \ref{eq4582}.
\end{theorem}

\begin{proof}\leanok Use the natural numbers $\N$ with operation $x \circ y := x$.
\end{proof}

\begin{theorem}[4 does not imply 43]\label{4_not_imply_43}\lean{Subgraph.Equation4_not_implies_Equation43}\leanok\uses{eq4,eq43} Definition \ref{eq4} does not imply Definition \ref{eq43}.
\end{theorem}

\begin{proof}\leanok Use the natural numbers $\N$ with operation $x \circ y := x$.
\end{proof}

\begin{theorem}[4582 does not imply 42]\label{4582_not_imply_42}\lean{Subgraph.Equation4582_not_implies_Equation42}\leanok\uses{eq4582,eq42} Definition \ref{eq4582} does not imply Definition \ref{eq42}.
\end{theorem}

\begin{proof}\leanok Use the natural numbers $\N$ with operation
$x \circ y$ equal to $1$ if $x=y=0$ and $2$ otherwise.
\end{proof}

\begin{theorem}[4582 does not imply 43]\label{4582_not_imply_43}\lean{Subgraph.Equation4582_not_implies_Equation43}\leanok\uses{eq4582,eq43} Definition \ref{eq4582} does not imply Definition \ref{eq43}.
\end{theorem}

\begin{proof}\leanok Use the natural numbers $\N$ with operation $x \circ y$ equal to $3$ if $x=1$ and $y=2$ and $4$ otherwise.
\end{proof}

\begin{theorem}[42 does not imply 43]\label{42_not_imply_43}\lean{Subgraph.Equation42_not_implies_Equation43}\leanok\uses{eq42,eq43} Definition \ref{eq42} does not imply Definition \ref{eq43}.
\end{theorem}

\begin{proof}\leanok Use the natural numbers $\N$ with operation $x \circ y := x$.
\end{proof}

\begin{theorem}[42 does not imply 4512]\label{42_not_imply_4512}\lean{Subgraph.Equation42_not_implies_Equation4512}\leanok\uses{eq42,eq4512} Definition \ref{eq42} does not imply Definition \ref{eq4512}.
\end{theorem}

\begin{proof}\leanok Use the natural numbers $\N$ with operation $x \circ y := x+1$.
\end{proof}

\begin{theorem}[43 does not imply 42]\label{43_not_imply_42}\lean{Subgraph.Equation43_not_implies_Equation42}\leanok\uses{eq43,eq42} Definition \ref{eq43} does not imply Definition \ref{eq42}.
\end{theorem}

\begin{proof}\leanok Use the natural numbers $\N$ with operation $x \circ y := x+y$.
\end{proof}

\begin{theorem}[43 does not imply 4512]\label{43_not_imply_4512}\lean{Subgraph.Equation43_not_implies_Equation4512}\leanok\uses{eq43,eq4512} Definition \ref{eq43} does not imply Definition \ref{eq4512}.
\end{theorem}

\begin{proof}\leanok Use the natural numbers $\N$ with operation $x \circ y := x \cdot y + 1$.
\end{proof}

\begin{theorem}[4513 does not imply 4522]\label{4513_not_imply_4522}\lean{Subgraph.Equation4513_not_implies_Equation4522}\leanok\uses{eq4513,eq4522} Definition \ref{eq4513} does not imply Definition \ref{eq4522}.
\end{theorem}

\begin{proof}\leanok Use the natural numbers $\N$ with operation $x \circ y$ equal to $1$ if $x=0$ and $y \leq 2$, $2$ if $x=0$ and $y>2$, and $x$ otherwise.
\end{proof}

\begin{theorem}[4512 does not imply 4513]\label{4512_not_imply_4513}\lean{Subgraph.Equation4512_not_implies_Equation4513}\leanok\uses{eq4512,eq4513} Definition \ref{eq4512} does not imply Definition \ref{eq4513}.
\end{theorem}

\begin{proof}\leanok Use the natural numbers $\N$ with operation $x \circ y := x + y$.
\end{proof}

\begin{theorem}[387 does not imply 42]\label{387_not_imply_42}\lean{Subgraph.Equation387_not_implies_Equation42}\leanok\uses{eq387,eq42} Definition \ref{eq387} does not imply Definition \ref{eq42}.
\end{theorem}

\begin{proof}\leanok Use the boolean type $\mathrm{Bool}$ with $x \circ y := x || y$.
\end{proof}

\begin{theorem}[43 does not imply 387]\label{43_not_imply_387}\lean{Subgraph.Equation43_not_implies_Equation387}\leanok\uses{eq43,eq387} Definition \ref{eq43} does not imply Definition \ref{eq387}.
\end{theorem}

\begin{proof}\leanok Use the natural numbers $\N$ with $x \circ y := x+y$.
\end{proof}

\begin{theorem}[387 does not imply 4512]\label{387_not_imply_4512}\lean{Subgraph.Equation387_not_implies_Equation4512}\leanok\uses{eq387,eq4512} Definition \ref{eq387} does not imply Definition \ref{eq4512}.
\end{theorem}

\begin{proof}\leanok Use the reals $\R$ with $x \circ y := (x+y)/2$.
\end{proof}

\begin{theorem}[3 does not imply 42]\label{3_not_imply_42}\lean{Subgraph.Equation3_not_implies_Equation42}\leanok\uses{eq3,eq42} Definition \ref{eq3} does not imply Definition \ref{eq42}.
\end{theorem}

\begin{proof}\leanok Use the natural numbers $\N$ with $x \circ y := y$.
\end{proof}

\begin{theorem}[3 does not imply 4512]\label{3_not_imply_4512}\lean{Subgraph.Equation3_not_implies_Equation4512}\leanok\uses{eq3,eq4512} Definition \ref{eq3} does not imply Definition \ref{eq4512}.
\end{theorem}

\begin{proof}\leanok Use the natural numbers $\N$ with $x \circ y$ equal to $x$ when $x=y$ and $x+1$ otherwise.
\end{proof}

\begin{theorem}[46 does not imply 3]\label{46_not_imply_3}\lean{Subgraph.Equation46_not_implies_Equation3}\leanok\uses{eq46,eq3} Definition \ref{eq46} does not imply Definition \ref{eq3}.
\end{theorem}

\begin{proof}\leanok Use the natural numbers $\N$ with $x \circ y := 0$.
\end{proof}

\begin{theorem}[43 does not imply 3]\label{43_not_imply_3}\lean{Subgraph.Equation43_not_implies_Equation3}\leanok\uses{eq43,eq3} Definition \ref{eq43} does not imply Definition \ref{eq3}.
\end{theorem}

\begin{proof}\leanok Use the natural numbers $\N$ with $x \circ y := x+y$.
\end{proof}

\begin{theorem}[4582 does not imply 46]\label{4582_not_imply_46}\lean{Subgraph.Equation4582_not_implies_Equation46}\leanok\uses{eq4582,eq46} Definition \ref{eq4582} does not imply Definition \ref{eq46}.
\end{theorem}

\begin{proof}\leanok Use $\{0,1,2\}$ with $x \circ y$ defined to equal $1$ when $x=y=2$ and $0$ otherwise.
\end{proof}
