\chapter{Infinite models}

In this chapter we consider non-implications which are refuted only on infinite models, as those are
more challenging to prove---they can't be proved by directly giving an operation table and checking
which laws it satisfies.

We note some selected laws of order more than $5$, used for such non-implications.

\begin{definition}[Equation 5105]
  \label{eq5105}\uses{magma-def}
  Equation 5105 is the law $0  \formaleq 1 \op (1 \op (1 \op (0 \op (2 \op 1))))$ (or the equation $x = y \op (y \op (y \op (x \op (z \op y))))$).
\end{definition}

This law of order $5$ was mentioned in \cite{Kisielewicz2}.

\begin{definition}[Equation 28393]
  \label{eq28393}\uses{magma-def}
  Equation 28393 is the law $0  \formaleq  (((0 \op 0) \op 0) \op 1) \op (0 \op 2)$ (or the equation $x = (((x \op x) \op x) \op y) \op (x \op z)$).
\end{definition}

This law of order $5$ was introduced by Kisielewicz \cite{Kisielewicz}.

\begin{definition}[Equation 374794]
  \lean{Equation374794}\leanok
  \label{eq374794}\uses{magma-def}
  Equation 374794 is the law $0  \formaleq  (((1 \op 1) \op 1) \op 0) \op ((1 \op 1) \op 2)$ (or the equation $x = (((y \op y) \op y) \op x) \op ((y \op y) \op z)$).
\end{definition}

This law of order $6$ was introduced by Kisielewicz \cite{Kisielewicz}.

The singleton or empty magma obeys all equational laws.  One can ask whether an equational law admits nontrivial finite or infinite models.  An \emph{Austin law} is a law which admits infinite models, but no nontrivial finite models.  Austin \cite{austin} established the first such law, namely the order $9$ law
$$ (((1 \op 1) \op 1) \op 0) \op (((1 \op 1) \op ((1 \op 1) \op 1)) \op 2) \formaleq 0.$$
A shorter Austin law of order $6$ was established in \cite{Kisielewicz}:

\begin{theorem}[Kisielewicz's first Austin law]
  \lean{InfModel.Finite.Equation374794_implies_Equation2,InfModel.Equation374794_not_implies_Equation2}\leanok
  \label{kis-thm}\uses{eq374794,eq2}
  Definition \ref{eq374794} is an Austin law.
\end{theorem}

\begin{proof} \leanok  First we show that every finite model of Definition \ref{eq374794} is trivial.  Write $y^2 := y \op y$ and $y^3 := y^2 \op y$.  For any $y,z$, introduce the functions ${f_y: x \mapsto y^3 \op x}$ and ${g_{yz}: x \mapsto x \op (y^2 \op z)}$.  Definition \ref{eq374794} says that $g_{yz}(f_y(x))=x$, hence by finiteness $g_{yz}=f_y^{-1}$, showing that $g_{yz}$ does not depend on the value of $z$.  Since
$$ f_y(y^2 \op z) = g_{yz}(y^3),$$
it follows that $f_y(y^2 \op z)=f_y(y^3)$ which by injectivity of $f_y$ implies that $z\mapsto y^2 \op z$ is a constant function (with $y$ fixed).  Substituting $y^2$ for $y$ shows that the same is true for $z\mapsto (y^2 \op y^2) \op z$, and since
$$ f_y(z) = (y^2 \op y) \op z = (y^2 \op y^2) \op z$$
we conclude that $f_y$ is also a constant function.  But this function is already known to be injective, thus there do not exist distinct elements in its domain, showing that the model must be trivial.

To construct an infinite model, consider the magma of positive integers $\Z^+$ with the operation $x \op y$ defined by
$$
x \op y = \begin{cases}
2^x, & y=x\\
3^y, & x=1,\ y\not=1\\
\min(j, 1), & x=3^j,\ y \neq x\\
1, & else
\end{cases}.
$$
Then $y \op y = 2^y$ and $(y \op y) \op y = 1$ for all $y$.  Moreover, $(y \op y) \op z$ is a power of two for all $y, z$, and $(1 \op x) \op w = x$ for all $x$ whenever $w$ is a power of two.  It follows that this magma is a model of Definition \ref{eq374794}.
\end{proof}

An even shorter law (order $5$) was obtained by the same author in a follow-up paper \cite{Kisielewicz2}:

\begin{theorem}[Kisielewicz's second Austin law]\label{kis-thm2}\uses{eq2} Definition \ref{eq28393} is an Austin law.
\end{theorem}

\begin{proof} Using the $y^2$ and $y^3$ notation as before, the law reads
\begin{equation}\label{kis2-law}
   x = (y^3 \op x) \op (y \op z).
  \end{equation}
In particular, for any $y$, the map $T_y \colon x \mapsto y^3 \op x$ is injective, hence bijective in a finite model $G$.  In particular we can find a function $f : G \to G$ such that $T_y f(y) = y^3$ for all $y$  Applying \eqref{kis2-law} with $x = f(y)$, we conclude
$$ T_y(y \op z) = y^3 \op (y \op z) = f(y) $$
and thus $y \op z$ is independent of $z$ by injectivity of $T_y$.  Thus, the left-hand side of \eqref{kis2-law} does not depend on $x$, and so the model is trivial.  This shows there are no non-trivial finite models.

To establish an infinite model, use $\N$ with $x \op y$ defined by requiring
$$ y \op y = 2^y; \quad 2^y \op y = 3^y$$
and
$$ 3^y \op x = 3^y 5^x$$
for $x \neq 3^y$, and
$$ (3^y 5^x) \op z = x$$
for $z \neq 3^y 5^x$.  Finally set
$$ 2^{3^y} \op z = 3^y$$
for $z \neq 3^y, 2^{3^y}$.  All other assignments of $\op$ may be made arbitrarily. It is then a routine matter to establish \eqref{kis2-law}.
\end{proof}

In that paper a computer search was also used to show that no law of order four or less is an Austin law.

An open question is whether Definition \ref{eq5105} is an Austin law.  We have the following partial result from \cite{Kisielewicz2}:

\begin{theorem}[Equation 5105 has no non-trivial finite models]\lean{InfModel.Finite.Equation5105_implies_Equation2}\leanok\label{5105-nontrivial}\uses{eq5105} Definition \ref{eq5105} has no non-trivial finite models.
\end{theorem}

\begin{proof} \leanok From Definition \ref{eq5105} we see that the map $w \mapsto y \op w$ is onto, hence injective in a finite model.  Using this injectivity four times in Definition \ref{eq5105}, we see that $z \op y$ does not depend on $z$, hence the expression
$x \op (z \op y)$ does not depend on $x$.  By Definition \ref{eq5105} again, this means that $x$ does not depend on $x$, which is absurd in a non-trivial model.
\end{proof}

We also have such a non-implication involving two laws of order $4$:
\begin{definition}[Equation 3994]
  \label{eq3994}\uses{magma-def}
  Equation 3994 is the law $0 \op 1 \formaleq (2 \op (0 \op 1)) \op 2$ (or the equation $x \op y = (z \op (x \op y)) \op z$).
\end{definition}

\begin{definition}[Equation 3588]
  \label{eq3588}\uses{magma-def}
  Equation 3588 is the law $0 \op 1 \formaleq 2 \op ((0 \op 1) \op 2)$ (or the equation $x \op y = z \op ((x \op y) \op z)$).
\end{definition}

\begin{theorem}\label{finite_imp_3994_3588_thm}\uses{eq3994,eq3588}
  \lean{InfModel.Finite.Equation3994_implies_Equation3588}\leanok
  All finite magmas which satisfy definition \ref{eq3994} also satisfy definition \ref{eq3588}
\end{theorem}

\begin{proof}\label{finite_imp_3994_3588} \leanok
  For a finite magma $M$, consider the set $S = \{x \op y | x, y \in M\}$.
  Now $f_z : x \mapsto z \op x$ and $g_z : x \mapsto x \op z$.
  They both map $S$ to $S$, and due to the hypothesis $g_z \op f_z$ is the identity on $S$,
  so because $S$ is finite $f_z$ and $g_z$ must be inverse bijections on it, and therefore
  they commute.
\end{proof}

\begin{theorem}\label{non_imp_3994_3588_thm}
  \lean{InfModel.Equation3994_not_implies_Equation3588}\leanok
  There exists a magma which satisfies \ref{eq3994} and not \ref{eq3588}.
\end{theorem}

\begin{proof} \leanok
  Consider $\mathbb{N}$, with $x \op y$ defined as $x \oplus y$ (bitwise XOR) if $x$ and $y$ are even,
  $y+2$ if only $y$ is even, $x \dot - 2$ if only $x$ is even, and $0$ if both are odd.
  Note that the range of the operation is the set of even naturals.
  Definition \ref{eq3994} is satisfied, because for even $z$ we get $z \oplus (x \op y) \oplus z = x \op y$
  and for odd $z$ we get $(x \op y) + 2 \dot - 2 = x \op y$.
  Setting $x = y = z = 1$, definition \ref{eq3588} isn't satisfied.
\end{proof}
