\chapter{General implications}

In this chapter we record some general implications between equational laws.

\begin{theorem}[Singleton law implies all other laws]\label{singleton-all}  The singleton law (Definition \ref{eq2}) implies all other laws.
\end{theorem}

\begin{proof} This is clear from substitution.
\end{proof}

\begin{theorem}[All laws imply the trivial law]\label{all-trivial}  All laws imply the trivial law (Definition \ref{eq1}).
\end{theorem}

\begin{proof} Trivial.
\end{proof}

Every law $E$ has a \emph{dual} $E^{\mathrm{op}}$, formed by replacing the magma operation $\circ$ with its opposite $\circ^{\mathrm{op}}:(x,y) \mapsto y \circ x$.  For instance, the opposite of the law $x \circ y = x \circ z$ is $y \circ x = z \circ x$.  A list of equations and their duals can be found \href{https://github.com/teorth/equational_theories/blob/main/data/dual_equations.md}{here}.  Of the 4694 equations under consideration, 84 are self-dual, leaving 2305 pairs of dual equations.

The implication graph has a duality symmetry:

\begin{theorem}[Duality]\label{duality}  If $E,F$ are equational laws, then $E$ implies $F$ if and only if $E^{\mathrm{op}}$ implies $F^{\mathrm{op}}$.
\end{theorem}

\begin{proof} This is because a magma $M$ obeys a law $E$ if and only if the opposite magma $M^{\mathrm{op}}$ obeys $E^{\mathrm{op}}$.
\end{proof}

Some equational laws can be ``diagonalized'':

\begin{theorem}[Diagonalization]\label{diag}  An equational law of the form
  \begin{equation}\label{prediag} F(x_1,\dots,x_n) = G(y_1,\dots,y_m),
  \end{equation}
  where $x_1,\dots,x_n$ and $y_1,\dots,y_m$ are distinct indeterminates, implies the diagonalized law
$$ F(x_1,\dots,x_n) = F(x'_1,\dots,x'_n).$$
In particular, if $G(y_1,\dots,y_m)$ can be viewed as a specialization of $F(x'_1,\dots,x'_n)$, then these two laws are equivalent.
\end{theorem}

\begin{proof}  From two applications of \eqref{prediag} one has
$$ F(x_1,\dots,x_n) = G(y_1,\dots,y_m)$$
and
$$ F(x'_1,\dots,x'_n) = G(y_1,\dots,y_m)$$
whence the claim.
\end{proof}

Thus for instance, Definition \ref{eq7} is equivalent to Definition \ref{eq2}.
