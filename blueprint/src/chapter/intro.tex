\chapter{Introduction}

\begin{definition}\label{magma-def}\lean{Magma}\leanok A Magma is a set $G$ equipped with a binary operation $\circ: G \times G \to G$.
\end{definition}

A \emph{law} is a equation involving a finite number of indeterminate variables and the operation $\circ$.  A magma $G$ then obeys that law if the equation holds for all possible choices of indeterminate variables in $G$.  For instance, the commutative law
$$ x \circ y = y \circ x$$
holds in a magma $G$ if and only if that magma is abelian.

We will be interested in seeing which laws imply which other laws, in the sense that magmas obeying the former law automatically obey the latter.  We will also be interested in \emph{anti-implications} showing that one law does \emph{not} imply another, by producing examples of magmas that obey the former law but not the latter.
