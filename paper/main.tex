\documentclass[12pt]{article}

% Packages
\usepackage{amsmath, amssymb, amsthm}
\usepackage{geometry}
\usepackage{hyperref}
\usepackage{graphicx}
\usepackage{enumitem}
\usepackage{color}
\usepackage{mathtools}
\usepackage{tikz}
\usepackage{mathrsfs}

% Page Setup
\geometry{letterpaper, margin=1in}
\setlength{\parindent}{0pt} % No indent for paragraphs
\setlength{\parskip}{1em}   % Spacing between paragraphs

% Theorem Styles
\newtheorem{theorem}{Theorem}[section]
\newtheorem{lemma}[theorem]{Lemma}
\newtheorem{proposition}[theorem]{Proposition}
\newtheorem{corollary}[theorem]{Corollary}
\theoremstyle{definition}
\newtheorem{definition}[theorem]{Definition}
\newtheorem{example}[theorem]{Example}
\newtheorem{remark}[theorem]{Remark}

% Commands
\newcommand{\R}{\mathbb{R}}
\newcommand{\C}{\mathbb{C}}
\newcommand{\N}{\mathbb{N}}
\newcommand{\Z}{\mathbb{Z}}
\newcommand{\Q}{\mathbb{Q}}
\newcommand{\F}{\mathbb{F}}
\newcommand{\eps}{\varepsilon}

% Title Information
\title{Paper Plan}
\author{Author Names (in Alphabetical Order)}
\date{\today}

\begin{document}

\maketitle

\tableofcontents

\section{Introduction}

\subsection{Magmas and Equational Laws}
Introduce the key definitions, and list some past results. Also, mention OEIS sequences for the number of equations with a given number of operations.

Discuss the state of the art of undecidability, particularly the question: Is the EquationX $\Rightarrow$ Equation Y problem for a single variable undecidable in general?

\subsection{Equational Theories Project}
Describe the initial aims and history of the project.

\section{Results}
While a large number of theoretically interesting results are not expected, some notable ones can be listed here with links to blueprints/Lean as necessary. Proofs can be deferred to the appendix.

\begin{itemize}
    \item A new short Austin pair: Equation 3944 implies Equation 3588 \cite{finite_magmas}, but not for infinite magmas \cite{infinite_magmas}.
\end{itemize}

\section{Mathematical Foundations}
This section covers topics like free magmas (including those relative to theories), a completeness theorem, and confluence (unique simplification).

\section{Formal Foundations}
Here we describe the Lean framework used to formalize the project, covering technical issues such as:

\begin{itemize}
    \item Magma operation symbol issues
    \item Syntax (`LawX`) versus semantics (`EquationX`)
    \item "Universe hell" issues
    \item Additional verification (axiom checking, Leanchecker, etc.)
    \item Use of the `conjecture` keyword
\end{itemize}

\subsection{Contributions to Mathlib}
None yet, but presumably, some of what we do will be uploadable and should be mentioned.

\section{Project Management}
Shreyas Srinivas and Pietro Monticone have volunteered to take the lead on this section.

Discuss topics such as:
\begin{itemize}
    \item Project generation from \href{https://github.com/pitmonticone/LeanProject}{template}
    \item Github issue management with \href{https://github.com/teorth/equational_theories/labels}{labels} and \href{https://github.com/users/teorth/projects/1}{task management dashboard}
    \item Continuous integration (builds, blueprint compilation, task status transition)
    \item Pre-push git hooks
    \item Use of Lean Zulip and polls
\end{itemize}

\subsection{Handling Scaling Issues}
Mention early human-managed efforts and the need for forethought in setting up a GitHub organizational structure. Discuss the use of transitive reduction to keep the Lean codebase manageable.

\subsection{Other Design Considerations}
Explain the meaning of "trusting Lean" in a large project and highlight human issues that may arise, tools for external checks, PR reviews, and good practices like branch protection.

\section{Finite Magmas and Other Sources of Counterexamples}
Describe various sources of example magmas, including finite and linear magmas, and their role in ruling out implications. Also, discuss the computational and memory efficiencies needed.

\section{Metatheorems}
List some notable metatheorems, including those that did not mature in time for deployment but may still be useful in the future.

\section{Automated Theorem Proving}
Describe the automated theorem provers used in the project (Z3, Vampire, egg, etc.) and performance statistics. Explore semi-automated vs. fully automated methods and how these were integrated into the project.

\section{AI-assisted Contributions}
Current contributions include Claude’s assistance with front-end coding, with potential for more as the project progresses.

\section{User Interface}
Describe visualizations and explorer tools used in the project.

\section{Statistics and Experiments}
Analyze the implication graph and discuss test sets of implication problems for benchmarking theorem provers. Challenge: How can one automatically assign a difficulty level to an implication?

\section{Data Management}
Describe how data was handled during the project and how it will be managed going forward.

\section{Reflections}
Include testimonies from participants and reflections on the project, discussing the balance between automation and human input.

\section{Conclusions and Future Directions}
Summarize insights and future directions for the project, including potential databases and interesting equational laws.

\section*{Acknowledgments}
Acknowledgments to the broader Lean Zulip community and smaller contributors not listed as authors.

\appendix
\section{Proofs of Theoretical Results}
Provide the interesting proofs mentioned in the results section, while routine proofs can refer to the blueprint or Lean.

\section{Author Contributions}
List author contributions, using CRediT categories. Elaborate on how these categories are interpreted and add affiliations and grant acknowledgments.

\bibliographystyle{plain}
\bibliography{references}

\end{document}